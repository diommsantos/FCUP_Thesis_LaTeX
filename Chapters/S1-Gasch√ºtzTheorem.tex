\section{Gaschütz Theorem}

\begin{definition}
    For any finite group $H$, $\phi_H(m)$ will denote the number of ordered $m$-tuples $(x_1,...,x_m)$ of elements of $H$ that generate $H$.
\end{definition}

The last Theorem of this section was first proved by Gaschütz in \cite{GaschützZENGP}. We present here an alternative proof from Roquette adapted from \cite[p.~360]{FriedFA}.

\begin{theorem}
    \label{S1:GT}
    Let $\theta : G \rightarrow H$ be a surjective homomorphism of finite groups with $d(G) \le m$. 
    Let $\mathbf{h} = (h_1, ..., h_m)$ be a tuple that generates $H$.
    Then there exists a tuple of generators $\mathbf{g}=(g_1, ..., g_m)$ of $G$ such that $\theta(g_i) = h_i$, $i = 1,...,m$.
    Moreover the cardinality of the set $$\left\{(g_1,...,g_m) \in G^m |\subgen{g_1,...,g_m} = G \text{ and } \theta(g_i) = h_i \right\}$$ is independent of the choice of $h_1, ..., h_m$.
\end{theorem}

\begin{proof}
    For each subgroup $C$ of $G$ satisfying $\theta(C) = H$ and all tuples $\mathbf{a} = (a_1,...,a_m)$ that generate $H$ denote the number of $e$-tuples $\mathbf{c} \in C^m$ that generate $C$ and satisfy $\theta(\mathbf{c}) = \mathbf{a}$ by $\varphi_C(\mathbf{a})$. 
    
    Let $\mathbf{a} = (a_1,...,a_m) \in H^m$ be such that $H = \subgen{a_1, ..., a_m}$ and $C$ a subgroup of $G$ satisfying $\theta(C) = H$.
    We prove by induction on $\card{C}$ that $\varphi_C(\mathbf{a})$ is independent of $\mathbf{a}$.

    Let $e = \frac{\card{C}}{\card{H}}$.
    We first claim that if for every subgroup $B$ of $C$, $\theta(B) \ne H$ we have $\varphi_C(\mathbf{a}) = e^m$. Since $\card{\theta|_{C}^{-1}(\{a_i\})} = \card{\ker \theta|_{C}} = \card{C}/\card{H} = e$ for all $i$ there are $e^m$ $\mathbf{c} = (c_1, ..., c_m) \in C^m$ tuples that satisfy $\theta(\mathbf{c}) = \mathbf{a}$. In particular since the subgroup $\subgen{c_1, ..., c_m}$ generated by any such tuple is a subgroup that satisfies $\theta(\subgen{c_1, ..., c_m}) = H$, by the hypothesis on $C$, it must be $C$. 
    
    Assume now that $\varphi_B(\mathbf{a})$ is independent of $\mathbf{a}$ for every proper subgroup $B$ of $C$ satisfying $\theta(B) = H$. Then there are exactly $e^m$ elements $\mathbf{b} \in C^m$ with $\theta(\mathbf{b}) = \mathbf{a}$. Each such $\mathbf{b}$ generates a subgroup $B$ of $C$ satisfying $\theta(B) = H$. Hence,
    $$
    e^m = \varphi_C(\mathbf{a}) + \sideset{}{}{\sum_{B < C}^{'}} \varphi_B(\mathbf{a})
    $$
    where $\sum_{}^{'}$ indicates a sum over groups with $\theta(B) = H$. By assumption, the $\sum_{}^{'}$ is independent of $\mathbf{a}$. Therefore, so is $\varphi_C(\mathbf{a})$.

    Now choose a tuple of generators $\mathbf{g'} = (g_1', ..., g_m')$ for $G$. 
    Then $\theta(\mathbf{g'}) = \mathbf{h'}$ generates $H$. By the preceding paragraph, $\varphi_G(\mathbf{h}) = \varphi_G(\mathbf{h'}) \ge 1$. 
    Consequently, $G$ has a tuple of generators $\mathbf{g} =(g_1, ..., g_m)$ such that $\theta(\mathbf{g}) = \mathbf{h}$. 
    The cardinality of 
    $$
    \left\{(g_1,...,g_m) \in G^m |\subgen{g_1,...,g_m} = G \text{ and } \theta(g_i) = h_i \right\}
    $$ 
    is precisely $\varphi_G(\mathbf{h})$ which is independent of the choice of $\mathbf{h}$.
\end{proof}

\begin{theorem}
    \label{GaschutzT}
    Let $N$ be a normal subgroup of a finite group $G$ and let $g_1, ... g_m \in G$ be such that $G = \subgen{g_1, ..., g_m, N}$. 
    If $d(G) \le m$, then there exists elements $u_1, ..., u_m$ of $N$ such that $G = \subgen{g_1u_1, ..., g_mu_m}$. 
    Moreover the cardinality of the set $$\left\{(u_1,...,u_m) \in N^m | G = \subgen{g_1u_1, ..., g_mu_m} \right\}$$ is independent of the choice of $g_1, ..., g_m$.
\end{theorem}
\begin{proof}
By applying Theorem \ref{S1:GT}, where $H$ is taken to be the quotient group $G/N$, $\theta \colon G \rightarrow G/N$ is the natural projection, and $h_i = g_iN$, we can find elements $z_1, ..., z_m \in G$ such that $\theta(z_i) = z_iN = g_iN$ and $\subgen{z_1, ..., z_m} = G$. Since $z_iN = g_iN$, we have $z_i = g_iu_i$ for a unique $u_i \in N$ for all $i$, which leads to the desired conclusion.
\end{proof}
