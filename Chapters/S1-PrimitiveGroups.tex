\section{Primitive Groups}

An extensive exposition of primitive groups is available on \cite{Ballester-BolinchesCFG}.

\begin{definition}
    Let $G$ be a group and $L$ a subgroup of $G$. The \textbf{core} of $L$, is the group $\core[G]{L} = \bigcap_{g \in G} gLg^{-1}$. 
    When the ambient group $G$ is clear from the context, we may drop the subscript and simply write $\core{L}$. 
\end{definition}

\begin{theorem}
    Let $G$ be a group and $L$ a subgroup of $G$. The group $\core{L}$ is the biggest normal subgroup of $G$ contained in $L$, i.e if $N \nsub G$ and $N \subseteq L$ then $N \subseteq \core{L}$.
\end{theorem}

\begin{proof}
    That $\core{L} \subseteq L$ is obvious since $\bigcap_{g \in G} gLg^{-1} \subseteq 1L1^{-1} = L$. Now let $N \nsub G$ and $N \subseteq L$. We have that for all $g \in G$, $N = gNg^{-1} \subseteq gLg^{-1}$ and thus it follows that 
    $$
    N = \bigcap_{g \in G} gNg^{-1} \subseteq \bigcap_{g \in G} gLg^{-1} = \core{L}.
    $$
\end{proof}

\begin{definition}
    A group $G$ is \textbf{primitive} if it has a maximal subgroup with trivial core.
\end{definition}