\section{The minimal normal subgroups of \texorpdfstring{$L_k$}{Lk}}

\begin{theorem}
    \label{nabmnsub}
    If $M$ is not abelian, any minimal normal subgroup of $L_k$ is of the form $\Mi$ for some $i$.
\end{theorem}

\begin{proof}
    Let $N$ be a minimal normal subgroup of $L_k$.
    Once again by Theorem \ref{hommnsub}, for each $1 \le i \le k$, $\pi_i(N)$ is either $1$ or $M$. We also have that $\pi_j(N) = M$ for some $j$ otherwise $N$ would be the trivial subgroup which is a contradiction.

    Let $j$ be such that $\pi_j(N) = M$. Now we will prove that for any $1 \le i \le k$ different of $j$, $\pi_i(N) = 1$ which gives us the result.
    
    Suppose on the contrary that exists some $1 \le i \le k$ different of $j$ such that $\pi_i(N) = M$.
    By Theorem \ref{mnsubsc} we obtain $[N, \Mi] = 1$. Since $M$ is not abelian we can choose elements $m_1, m_2 \in M$ such that $m_1m_2 \ne m_2m_1$. Besides we have by hypothesis that $m_1 = \pi_i(n_1)$ and $m_2 = \pi_i(n_2)$ for some $n_1 \in N$ and some  $n_2 \in M_i$. It thus follows that:
    \begin{align*}
       m_1m_2 &= \pi_i(n_1)\pi_i(n_2) \\
              &= \pi_i(n_1n_2) \\
              &= \pi_i(n_2n_1) \text{, by } [N, \Mi] = 1 \\
              &= \pi_i(n_2)\pi_i(n_1) = m_2m_1 \text{, a contradiction.}
    \end{align*}

\end{proof}

With the previous theorem, the minimal normal subgroups in the case where $M$ is non abelian are fully characterized.
What can we say about the minimal normal subgroups of $L_k$ in the abelian case? The next theorems provide us some nice properties.

\begin{theorem}
    \label{abmnsub}
    Let $N$ be a minimal normal subgroup of $L_k$. Then $N$ has order $\card{M}$. 
    
    %Furthermore if for some $1 \le i \le k$, $\pi_i(N) = M$ and an element $n$ of $N$ has $i$-th coordinate equal to $1$ then all its coordinates are $1$, that is if $\pi_i(n) = 1$ for some $1 \le n \le k$ then $n = 1$.
\end{theorem}

\begin{proof}
    We have once more that $\pi_i(N) = M$ for some $1 \le i \le k$. Considering the appropriate restriction of $\pi_i$, by the First Isomorphism Theorem we get that $\card{N} = \card{M} \cdot \card{\ker{\pi_i|_{N}}}$ which implies $\card{N} \geq \card{M}$.

    Let us assume that $\card{N} > \card{M}$. Then there is some $n \in \ker{\pi_i|_{N}}$ with not all coordinates $1$ (obviously the $i$-th coordinate must be $1$).

    Consider $\subgen{n}_{L_k}$ contained in $\ker{\pi_i|_{N}}$. Since $n \in N$, and $\subgen{n}_{L_k}$ is the smallest normal subgroup generated by $n$ we must have $\subgen{n}_{L_k} \subseteq N$. Since $\pi_i(\subgen{n}_{L_k}) = \subgen{\pi_i(n)}_{L_k} = 1$ and not all elements of $N$ are in $\ker{\pi_i|_{N}}$ we obtain that $\subgen{n}_{L_k}$ is a non-trivial group strictly contained in $N$. This is a contradiction since $N$ is a minimal normal subgroup.
\end{proof}

\begin{definition}
    Given $l \in L$ and a positive integer $q$, let $\dl$ denote the element of $\diag{L^q}$ with all coordinates equal to $l$.
\end{definition}

The previous definition depends on which power of the group $L$ we are considering, and often the only way to decide the ambient group is through context, but what we lose in formal rigor we gain in readability. Such an example of improved readability is the statement of the next theorem.

\begin{theorem}
    \label{cplNsoc}
    Let $N$ be a minimal normal subgroup of $L_k$. Then there exists a complement $C \nsub L_k$ of $N$ in $\soc{L_k}$ and an isomorphism $\phi \colon C \rightarrow M^{k-1}$ that satisfies $\phi(c^{\dl}) = \phi(c)^{\dl}$ for any $c \in C$ and $l \in L$.
\end{theorem}

\begin{proof}
    We have once more that $\pi_i(N) = M$ for some $1 \le i \le k$. 
    
    Let $C = \Mi[1]...\Mi[i-1]\Mi[i+1]...\Mi[k]$. We claim that $N \cap C = 1$. Let us note first that any element of $N \cap C \subseteq C$ has $i$-th coordinate $1$. Thus $N \cap C \nsub L_k$ is strictly contained in the minimal normal subgroup $N$. So it must be $1$.

    From the Second Isomorphism Theorem we get that $\card{NC}/\card{N} = \card{C} / \card{N \cap C}$.
    We obtain $\card{NC} = \card{N}\card{C}\cdot 1$ and since $\card{N} = \card{M}$ by Theorem \ref{abmnsub}, $\card{NC} = \card{M}\card{C} = \card{M}^k = \card{\soc{L_k}}$.
    Obviously $NC \subseteq \soc{L_k}$ as $N,C \subseteq \soc{L_k}$. We thus conclude that $NC = \soc{L_k}$.
    

    The expected isomorphism
    \begin{align*}
        \phi \colon C = M \times ... \times M \times &1 \times M \times ... \times M \longrightarrow M^{k-1} \\
        (m_1,...,m_i,&1,m_{i+1},...,m_{k}) \mapsto (m_1,...,m_i,m_{i+1},...,m_{k})
    \end{align*}
    works. 
    
    The property is easily verified since for any $l \in L$, $(m_1,...,m_i,1,m_{i+1},...,m_{k}) \in C$:
    \begin{align*}
        &\phi(lm_1l^{-1},...,lm_il^{-1},l1l^{-1},lm_{i+1}l^{-1},...,lm_{k}l^{-1}) &= \\
        &(lm_1l^{-1},...,lm_il^{-1},lm_{i+1}l^{-1},...,lm_{k}l^{-1}) &= \\
        &(l,...,l)(m_1,...,m_i,m_{i+1},...,m_{k})(l^{-1},...,l^{-1}) &= \\ 
        &(l,...,l)\phi(m_1,...,m_i,1,m_{i+1},...,m_{k})(l,...,l)^{-1}
    \end{align*}
\end{proof}
