\chapter*{Introduction}
\addcontentsline{toc}{section}{Introduction}
One of the questions in finite group theory is to determine the minimal number of generators of a finite group.

For a finite group $H$, the minimal number of generators $d(H) = min\left\{\card{X}|\subgen{X} = G \right\}$ always exists. This is so because $H$ is always generated by itself, a finite set. On the other hand there are infinite groups that don't have a minimal number of generators. One such example is $\mathbb{Z}^{\mathbb{N}}$, the infinite direct product of copies of $\mathbb{Z}$.

It is not generally true that given a group $H$ and a subgroup $K \le H$, $d(K) \le d(H)$. 
In fact, the evidence suggests that there is very little we can say in general about the relationship between $d(H)$ and $d(K)$ for some subgroup $K$ of $H$. By Cayley's Theorem \cite[p.~52]{RotmanITG}, every finite group can be embedded in a symmetric group $S_n$, given a large enough integer $n$. It is also well known that $d(S_n) = 2$ \cite[p.~24]{RotmanITG}. On the other hand there are finite groups with any minimal number of generators. The group $(\mathbb{Z}_2)^d$, the direct product of $d$ copies of the additive group of integers modulo $2$, is generated by $d$ elements for any positive integer $d$. Thus, any result about the minimal number of generators of subgroups has to account for the fact that any finite group can be embedded in a finite group generated by two elements.

On the other hand, it is easily verifiable that for any $N \nsub H$, $d(H/N) \le d(H)$. 
This is so as the generators $h_1,...,h_n$ of $H$, induce generators of $H/N$, namely $h_1N, ..., h_nN$.


It has been shown \cite{AschbacherSAFCG} that two generators are sufficient to generate any simple group.
With our current understanding, we can already break down the problem and make meaningful progress in addressing the issue for generic finite groups.

In fact for a generic finite group $H$, either $d(H) > d(H/K)$ for all non-trivial subgroups $K \nsub H$ or there is some non-trivial $K \nsub H$ such that $d(H) = d(H/K)$. In the second case the problem is reduced to that of determining the minimal number of generators of $H/K$, usually an easier task since $H/K$ has a lesser order than that of $H$. Now the group $H/K$ can again have a non-trivial normal subgroup $L/K$ such that $d(H/K) = d((H/K)/(L/K))$ and if that is the case the problem can again be simplified. Proceeding in this manner, the problem of determining the minimal number of generators of a generic finite group $H$ can be reduced to the problem of finding the minimal number of generators of finite groups $K$ which are generated by more elements than any of its proper non-trivial quotients i.e the first case.

Spectacular results regarding this problem were presented in \cite{DallaVoltaFGNMGAPQ}. These results rely on advanced group theory concepts and, although extremely valuable, have some omissions in their explanation, which this dissertation aims to fill.
