\section{G-homomorphisms}

As a reminder we enunciate here the definition of a $G$-set.

\begin{definition}
    \cite[p.~55]{RotmanITG}
    If $X$ is a set and $G$ is a group, then $X$ is a \textbf{$G$-set} if there is a function $\alpha : G \times X \to X$ (called an \textbf{action}), denoted by $\alpha(g, x) \mapsto gx$, such that:
    \begin{enumerate}
    \item $\alpha(e, x) = x$ for all $x \in X$;
    \item $\alpha(g, \alpha(h, x)) = \alpha(gh, x)$ for all $g, h \in G$ and $x \in X$.
    \end{enumerate}
    One also says that \textbf{$G$ acts on $X$}. If $|X| = n$, then $n$ is called the \textbf{degree} of the $G$-set $X$.
\end{definition}
    

A more profound exposition of $G$-homomorphisms is available on \cite[p.~260]{RotmanITG}, but for our purposes just the definition suffices.

\begin{definition}
    If $X$ and $Y$ are $G$-sets, a function $f: X \rightarrow Y$ is a \textbf{$G$-homomorphism} if $f(g \cdot x) = g \cdot f(x)$ for all $x \in X$ and $g \in G$. If $f$ is also a bijection, then $f$ is called a \textbf{$G$-isomorphism}.

\end{definition}