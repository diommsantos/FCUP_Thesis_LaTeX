\section{G-homomorphisms}

A more profound exposition of $G$-homomorphisms is available on \cite[p.~260]{RotmanITG}, but for our purposes just the definition suffices.

\begin{definition}
    If $X$ and $Y$ are $G$-sets, a function $f: X \rightarrow Y$ is a \textbf{$G$-homomorphism} if $f(g \cdot x) = g \cdot f(x)$ for all $x \in X$ and $g \in G$. If $f$ is also a bijection, then $f$ is called a \textbf{$G$-isomorphism}.

\end{definition}