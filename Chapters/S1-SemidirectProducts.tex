\section{Semidirect Products}

\begin{definition}
    Let $K$ be a subgroup of a group $G$. A subgroup $Q \subseteq G$ is a \textbf{complement} of $K$ in $G$ if $K \cap Q = \{1\}$ and $KQ = G$.
\end{definition}

\begin{definition}
    A group $G$ is a \textbf{semidirect product} of $K$ by $Q$, denoted by $G = K \rtimes Q$, if $K \nsub G$ and $K$ has a complement $Q' \cong Q$.

\end{definition}

The next theorem can be considered as transitivity for semidirect products.

\begin{theorem}
\label{smdptrans}
Let $G \le H \le K$ be groups. Suppose that $G$ is complemented in $H$, its complement is normal in $K$ and $H$ is complemented in $K$ then $G$ is complemented in $K$.
\end{theorem}

\begin{proof}
    \label{S1:SPE}
    Let $H'$ be the complement of $H$ in $K$ and $G'$ the complement of $G$ in $H$. Since by hypothesis $G'$ is normal in $K$ we have that $H'G'$ is a group. Also by hypothesis we have that:
    $K = H'H =H'(G'G) = (H'G')G$.
    Furthermore $H'G' \cap G = 1$ because if $g \in G \cap H'G'$ then $g = h'g'$ for some $h' \in H'$ and $g' \in G'$. We have that $gg'^{-1} = h' \in H' \cap GG' = H' \cap H = 1$ and so it follows that $h' = 1$ and that $g = g' \in G \cap G' = 1$.
\end{proof}

\begin{theorem}
    Let $G$ be a group, $C_1, \ldots , C_n$ be a  finite sequence of subgroups of $G$ and $A \le G$ any subgroup. If $C_1\ldots C_n \cap A = 1$
    $$
    \bigcap_{l = 1}^{n}(AC_l) = A(\bigcap_{l = 1}^{n}C_l).
    $$
\end{theorem}

\begin{proof}
    We will do this proof by induction on $n$.
    Let $n = 2$. 
    
    We will start by proving $AC_1 \cap AC_2 \subseteq A(C_1 \cap C_2)$.
    We have that $x \in AC_1 \cap AC_2 \iff a_1c_1 = x = a_2c_2$ for some $a_1, a_2 \in A$, $c_1 \in C_1$ and $c_2 \in C_2$. From this follows that $a_2^{-1}a_1 = c_2c_1^{-1} \in A \cap C_1C_2 = 1$. Thus we conclude that $a_1 = a_2$ and $c_1 = c_2 \in C_1 \cap C_2$, that is $x = ac_1 \in A(C_1 \cap C_2)$ and the first inclusion is thus proved. 

    The other inclusion is trivial.

    Let us assume now that the result holds for $n$. 
    Then by the induction hypothesis,
    $$
    \bigcap_{i = 1}^{n}(AC_i) \cap AC_{n+1} = A(\bigcap_{i = 1}^{n}C_i) \cap AC_{n+1}.
    $$
    The rest of the proof is now analogous to the case $n = 2$ with $C_1 = \bigcap_{i = 1}^{n}C_i$ and $C_2 = C_{n+1}$.
\end{proof}