\section{Free Groups}

\begin{definition}
    Let $X$ be a subset of a group $F$. We say that $F$ is a \textbf{free group with basis $X$} if, for every group $G$ and every function $f \colon X \rightarrow G$, there exists a unique homomorphism $\varphi \colon F \rightarrow G$ extending $f$ ($\varphi |_X = f$). In other words denoting by $i \colon X \rightarrow F$ the inclusion, $F$ is a \textbf{free group with basis $X$} if the following diagram commutes 
    % https://q.uiver.app/#q=WzAsMyxbMCwwLCJYIl0sWzEsMCwiRyJdLFswLDEsIkYiXSxbMCwxLCJmIl0sWzAsMiwiaSIsMl0sWzIsMSwiXFx2YXJwaGkiLDJdXQ==
    \[\begin{tikzcd}
        X & G \\
        F
        \arrow["f", from=1-1, to=1-2]
        \arrow["i"', from=1-1, to=2-1]
        \arrow["\varphi"', from=2-1, to=1-2]
    \end{tikzcd}.\]
\end{definition}

\begin{theorem}
    \cite[p.~344]{RotmanITG}
    Given a set $X$, there exists a free group with basis $X$.
\end{theorem}

\begin{theorem}
    \cite[p.~348]{RotmanITG}
    Let $F$ and $G$ be free groups with bases $X$ and $Y$, respectively. 
    Then $F \cong G$ if and only if $\card{X} = \card{Y}$. 
\end{theorem}

Taking $G$ as $F$ it easily follows from the last theorem that any basis $X$ of a free group $F$ has the same number of elements.

\begin{definition}
    The \textbf{rank} of a free group $F$, denoted by $\mathbf{rank(F)}$, is the number of elements in a basis of $F$. 
\end{definition}