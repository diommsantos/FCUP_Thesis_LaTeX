\pagebreak

\section{Minimal Normal Subgroups}

\begin{definition}
A normal subgroup $M$ of a group $G$ is said to be a \textbf{minimal normal subgroup} if it is non-trivial and it doesn't contain any proper non-trivial normal subgroup. That is $M$ is a \textbf{minimal normal subgroup} if $M \ne 1$ and there is no normal subgroup $K$ of $G$ such that $1 < K < M$. 
\end{definition}

There are groups without minimal normal subgroups. One example is the additive group $\mathbb{Z}$. Any subgroup of $\mathbb{Z}$ (all subgroups of $\mathbb{Z}$ are normal) is of the form $m\mathbb{Z}$ for some positive integer $m$. Taking the subgroup $2m\mathbb{Z}$ we get a non-trivial normal subgroup contained in $m\mathbb{Z}$.

On the other hand minimal normal subgroups always exist for non-trivial finite groups. 
Let us suppose there is a non-trivial finite group $H$ without a minimal normal subgroup. Then we can construct the following chain of normal subgroups of $H$,
$$
H > M_1 > M_2 > ...
$$
were each subgroup $M$ is strictly contained in the one before. Since none of this groups by assumption can be a minimal normal subgroup we can prolong this chain forever and thus arise at a contradiction on the finiteness of $H$.

\begin{theorem}
    \label{hommnsub}
    Let $G$ and $H$ be groups.
    Given a surjective homomorphism $\alpha\colon G \rightarrow H$ and $M$ a minimal normal subgroup of $G$, $\alpha(M)$ is either a minimal normal subgroup of $H$ or $1$.
\end{theorem}
\begin{proof}
    Let us assume that $\alpha(M)$ is neither a minimal normal subgroup of $H$ neither $1$.

    Since $\alpha$ is surjective, we have that $\alpha(M)$ is normal in $H$, and by assumption there is a normal subgroup $N$ strictly contained in $\alpha(M)$.
    
    Obviously $\alpha^{-1}(N)$ is normal in $G$. Considering now the normal subgroup $\alpha^{-1}(N) \cap M$ we see that it is non-trivial and strictly contained in $M$, contradicting the minimality of $M$.
\end{proof}

\begin{definition}
    Let $x$ and $y$ be elements of a group $G$. We define the \textbf{commutator of $x$ and $y$} as 
    $$
    [x,y] = hkh^{-1}k^{-1}.
    $$
\end{definition}

Let us notice that the commutator of two elements is the identity if and only if they commute.

Similarly we can define the commutator of two subgroups.

\begin{definition}
    \label{S1:groupcommutator}
    Let $G$ be a group. If $H, K \le G$, then
    $$
    [H,K] = \subgen{\left\{ hkh^{-1}k^{-1} | h \in H \text{ and } k \in K \right\}}.
    $$ 
\end{definition}

Likewise if $H,K \le G$, $[H,K] = 1$ if and only if all the elements of $H$ commute with all the elements of $K$.
The set $\left\{ hkh^{-1}k^{-1} | h \in H \text{ and } k \in K \right\}$ is not necessarily a subgroup, an example is provided in \cite{CassidyPCANACAE}, hence we take the smallest subgroup generated by the commutators in the definition.

\begin{theorem}
    \label{mnsubsc}
    If $M_1$ and $M_2$ are distinct minimal normal subgroups then they centralize each other.
\end{theorem}

\begin{proof}
    We have that $[ M_1, M_2] \le M_1 \cap M_2$ and as $M_1 \ne M_2$, $M_1 \cap M_2 = 1$.
\end{proof}