\pagebreak
\section{Nilpotent Groups}

We enumerate here some well known results about nilpotent groups that will be used throughout. The proofs of these results can be found in \cite{RotmanITG}, and are omitted here since the exposition of this subject benefits immensely from a more comprehensive treatment.

\begin{definition}
    Let $G$ be a group. The groups $\gamma_i(G)$ are defined by induction as:
    $$
    \gamma_1(G) = G; \quad \gamma_{i+1} = \left[\gamma_i(G), G\right].
    $$ 
\end{definition}

\begin{definition}
    A group $G$ is \textbf{nilpotent} if there is an integer $c$ such that $\gamma_{c+1}(G) = 1$. 
\end{definition}

\begin{theorem}
    \cite[p.~116]{RotmanITG}
    A finite group $H$ is nilpotent if and only if it is the direct product of its Sylow subgroups.
\end{theorem}

\begin{theorem}
    \cite[p.~117]{RotmanITG}
    \label{S1NG:maxsub}
    If $H$ is a finite $p$-group, then every maximal subgroup of $H$ is normal and has index $p$.
\end{theorem}

