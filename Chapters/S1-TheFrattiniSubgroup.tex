\section{The Frattini Subgroup}

Throughout this section, unless explicitly said otherwise $H$ will always denote a finite group. 

For non-trivial finite groups, maximal subgroups always exist. Assuming otherwise, let $H$ be a non-trivial finite group without maximal subgroups. Then we can construct the sequence
$1 < K_1 < K_2 \ldots $
where each group $K$ is strictly contained in the one before. Since none of this groups by assumption can be a maximal subgroup of $H$ we can can prolong this sequence forever and thus arise at a contradiction on the finiteness of $H$.

On the contrary maximal subgroups might not exist for infinite groups and an example is provided in \cite[p.~123]{RotmanITG}.

\begin{definition}
    The \textbf{Frattini subgroup of $H$}, denoted by \textbf{$\Phi(H)$}, is the intersection of all maximal subgroups of $H$ if $H$ has a maximal subgroup, otherwise $\Phi(H)$ is set to be $H$. 
\end{definition}

\begin{theorem}
    \cite[Theorem 5.47]{RotmanITG}
    The Frattini subgroup of a nontrivial finite group $H$ is the set of all nongenerators, that is the set of those elements $h \in H$ such that if $H = \subgen{Y, h}$ then $H = \subgen{Y}$ for any set $Y \subseteq H$.
\end{theorem}

\begin{proof}
    Let $h \in \Phi(H)$ and let $Y \subseteq H$ be such that $\subgen{Y, h} = H$. If $\subgen{Y} \ne H$, we have that $\subgen{Y} \le M$ for some maximal subgroup $M$ of $H$. Since $h \in \Phi(H)$, in particular $h \in M$.  But this implies that $\subgen{Y,h} \le M \ne H$, a contradiction.

    Conversely let $z$ be a nongenerator and $M$ a maximal subgroup of $H$. If $z \notin M$ then $H = \subgen{z, M} = \subgen{M} = M$, which is a contradiction.
\end{proof}

\begin{theorem}
    \cite[Theorem 5.48]{RotmanITG}
    \label{fratpgroup}
    Let $H$ be a finite $p$-group. Then:
    \begin{enumerate}
        \item $\Phi(H) = H'H^p$ where $H^p$ is the subgroup of $H$ generated by all $p$-th powers,
        \item $H/\Phi(H) \cong (\mathbb{Z}_p)^q$ for some positive integer $q$.
    \end{enumerate}
\end{theorem}

\begin{proof}
    \begin{enumerate}
        \item Let $M$ be a maximal subgroup of $H$. According to Theorem \ref{S1NG:maxsub}, $M$ is a normal subgroup of $H$ with index $p$. 
        Hence, the quotient group $H/M$ is abelian, implying that the commutator subgroup $H'$ is contained in $M$. 
        Furthermore, $H/M$ has exponent $p$, meaning every element of $H$ raised to the $p$-th power lies in $M$. 
        Thus $H'H^p \le \Phi(H)$.

        To show the reverse inclusion, consider the quotient group $H/H'H^p$.
        It is an abelian group of exponent $p$, hence isomorphic to $(\mathbb{Z}_p)^q$ for some positive integer $q$ and thus can be regarded as a vector space over the field $\mathbb{F}_p$.
        It is evident that its Frattini subgroup is trivial.
        Now, if we have $N \nsub H$ such that $N \le \Phi(H)$, it can be verified easily that $\Phi(H)$ is in the preimage (under the natural quotient map $\pi$) of $\Phi(H/N)$, as maximal subgroups correspond.
        Thus $\Phi(H) \subseteq \pi^{-1}(\Phi(H/(H'H^p))) = \pi^{-1}(1) \subseteq H'H^p$ and we conclude that $\Phi(H) = H'H^p$.

        \item Since $H' \le H'H^p = \Phi(H)$, $H/\Phi(H)$ is abelian. Furthermore since $H^p \le \Phi(H)$, $H/\Phi(H)$ has exponent $p$.
        Thus $H \cong (\mathbb{Z}_p)^q$ for some positive integer $q$.

    \end{enumerate}
\end{proof}

\begin{theorem}
\label{th:fratgen}
Let $H$ be a finite group. Then $d(H) = d(H/\Phi(H))$.
\end{theorem}
\begin{proof}
    Let $d = d(H/\Phi(H))$ and suppose $H/\Phi(H) = \subgen{g_1\Phi(H),\ldots ,g_d\Phi(H)}$. 
    Then $H = \subgen{g_1,\ldots ,g_d,\Phi(H)}$ and since $\Phi(H)$ is the set of non-generators of $H$, that is the set of those elements $h \in H$ such that if $H = \subgen{Y, h}$ then $H = \subgen{Y}$ for any set $Y \subseteq H$, the result follows.
\end{proof}
