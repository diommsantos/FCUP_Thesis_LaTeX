\section{The function \texorpdfstring{$f$}{f}}

We are now in conditions to define the function $f$. This function will play a key role in finding out the minimal number of generators of a finite group.

Before providing the definition, let us remember that we are assuming $L$ to always denote a finite group with a unique minimal normal subgroup $M$, complemented if $M$ is abelian. Furthermore we will define $L_0$ as $L/M$.

\begin{definition}
    Given a group $L$ we define $f(L, m) = k+1$ if and only if $d(L_k) = m < d(L_{k+1})$. When $L$ can be identified from the context, we denote $f(L,m)$ as $f(m)$.
\end{definition}

The function $f$ gives us the integer $k+1$ for which $d(L_{k+1}) = m + 1$ is bigger than any $d(L_q)$ for $q < k+1$ (we are implicitly using Theorem \ref{S2:bounddLk} in this claim). 

We claim that any proper quotient of $L_{k+1}$ has a minimal number of generators smaller than or equal to $m$. 
Let $N$ be a non-trivial normal subgroup of $L_{k+1}$. 
There is a minimal normal subgroup $M$ contained in $N$ and  thus by Theorem \ref{S2:QLkmnsub} we obtain that $d(L_{k+1} / M) = d(L_k) \le m$. 
Now using the Third Isomorphism Theorem and the fact that the minimal number of generators of a quotient is always smaller or equal than the ambient group we obtain
$$
d(L_{k+1}/N) = d(\frac{L_{k+1}/M}{N/M}) \le d(L_{k+1} / M) = d(L_k) = m.
$$
Thus the function $f$ gives us \textit{the integer $k+1$ for which any proper quotient of $L_{k+1}$ has minimal number of generators smaller or equal to $m$ but $d(L_{k+1}) > m$.}

In light of this new characterization of the function $f$, our reason for our definition of $L_0$ is now justified, that is $f(m) = 1$ iff $d(L_0) = d(L/M) < d(L)$. 

Let us notice that this function of course depends on the group $L$ being considered and that the domain of this function is the positive integers. 

Furthermore this function is well defined. Let $m_1 = m_2$ be two positive integers. Then since $d(L_k)_{k \in \mathbb{N}}$ is non-decreasing by Theorem \ref{S2:nondecLk} and unlimited by Theorem \ref{S2:undLk} there is some positive integer $k$ such that $d(L_k) = m_1 = m_2 < d(L_{k+1})$. Since the sequence is non-decreasing this number $k$ must be unique and thus $f(L,m_1) = k+1 = f(L,m_2)$.