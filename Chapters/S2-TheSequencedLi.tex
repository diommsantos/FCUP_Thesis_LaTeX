\section{The Sequence \texorpdfstring{$d(L_i)_{i \in \mathbb{N}}$}{dLiN}}

The sequence $d(L^i)_{i \in \mathbb{N}}$ is called the \textit{growth sequence} and has been studied in \cite{WiegoldGSFG, WiegoldGSFGII, WiegoldGSFGIII, WiegoldGSFGIV}. Given the importance that the groups $L_k$ assume in the study of the minimal number of generators of finite groups, in this section we will study some properties of the sequence $d(L_i)_{i \in \mathbb{N}}$. 

\begin{theorem}
    $d(L_{k+1}) \le d(L_k) + 1$    
\end{theorem}
\begin{proof}
    Let $d(L_k) = d$, then there are $l_1, l_2, ... l_d \in L_k$ such that $\subgen{l_1, l_2, ... l_d} = L_k$. By abuse of notation, given $l \in L_k$ let $\ell \in L_{k+1}$ be the $k+1$-tuples whose first $k$ coordinates are the coordinates of $l$ and whose last two coordinates are equal, that is $\pi_k(\ell) = \pi_{k+1}(\ell)$. 
        
    Let $m \in 1 \times ... \times 1 \times M \le L_{k+1}$ and consider
    $$H = \subgen{m^\ell | l \in L_k} \le L_{k+1}.$$
    For all $i \in \left\{1, ..., k \right\}$, $\pi_i(H) = 1$, because
    $$\pi_i(\subgen{m^\ell | l \in L_k}) = \subgen{\pi_i(m)^{\pi_i(\ell)} | l \in L_k} = \subgen{1^{\pi_i(\ell)} | l \in L_k} = 1.$$ 
    And $\pi_{k+1}(H) = M$ since
    $$\pi_{k+1}(\subgen{m^\ell | l \in L_k}) = \subgen{\pi_{k+1}(m)^{\pi_{k+1}(\ell)} | l \in L_k} = \subgen{\pi_{k+1}(m)}_L,$$
    where the last inequality follows since $\pi_{k+1}(l_1) = l_1, \pi_{k+1}(l_2) = l_2, ..., \pi_{k+1}(l_d) = l_d$ and $\subgen{l_1, l_2, ... l_d} = L_k$. We thus have that $\pi_{k+1}(H)$ is a nontrivial normal subgroup of $L$ that is contained in $M$, thus $\pi_{k+1}(H) = M$ due to the minimality of $M$. We thus have that $H = 1 \times ... \times 1 \times M$.
    Since $L_{k+1} = \left\{ \ell | l \in L_{k+1} \right\}(1 \times ... \times 1 \times M) = \left\{ \ell | l \in L_{k+1} \right\} H = \subgen{\ell_1, \ell_2, ... , \ell_d, m}.$
    
\end{proof}

\begin{theorem}
    The sequence $d(L_1),...,d(L_k), ....$ is unlimited.
\end{theorem}

\begin{proof}
    Suppose for the sake of obtaining a contradiction that there exists a natural number $m$ such that $d(L_k)<m$ for all $k \in \mathbb{N}$.
    Let $F$ be the free group of rank $m$ and $K$ be the set of positive integers $k$ such that there exists a surjective homomorphism from $F$ to $L_k$. 
    It will be proved that $K$ is a finite set, a contradiction from which the result will follow.

    Let $k > 1 \in K$, then there exists a surjective homomorphism $\Psi : F \rightarrow L_k$. For $1 \le i \le k$, let $\gamma_i = \pi_i \circ \Psi : F \rightarrow L_k \rightarrow L$ and let $\pi : L \rightarrow L/M$ be the natural projection. For all $x \in F$, $\Psi(x) = (l_1,...,l_k)$ for some $(l_1,...,l_k) \in L_k$ and thus:
    \begin{align*}
        &Ml_1 = Ml_2 = ... = Ml_k \text{ and } \\
        &\pi\gamma_1(x) = Ml_1, .... , \pi\gamma_k(x) = Ml_k \\
        \implies &\pi\gamma_1 = \pi\gamma_2 = ... = \pi\gamma_k.
    \end{align*}
    Let $i_1, i_2 \in \left\{ 1, ..., k \right\}$ and let $(m_1,..., m_k) \in M_k \le L_k$ with $m_{i_1} = 1, m_{i_2} \ne 1.$ Since $\Psi : F \rightarrow L_k$ is surjective there exists an $x \in F$ such that $\Psi(x) = (m_1,..., m_k)$. 
    We then have that $\gamma_{i_1}(x) = m_{i_1} = 1$ and $\gamma_{i_2}(x) = m_{i2} \ne 1$, and thus $x \in \ker \gamma_{i_1}$ and $x \notin \ker \gamma_{i_2}$ and hence $\ker \gamma_{i_1} \ne \ker \gamma_{i_2}$. We thus have that $\card{\left\{ \ker \gamma_{i_1}, ..., \ker \gamma_{i_k} \right\}} = k$. 
    
    By Theorem \ref{PHallT}, taking $\beta$ as $\pi\gamma_i$ for any $i$ (it was proved that this functions were all equal) and seeing $\left\{ \ker \gamma_{i_1}, ..., \ker \gamma_{i_k} \right\}$ as a subset of the appropriate $R$, the set of normal subgroups $N$ of $F$ arising as kernels of those homomorphisms of $F$ onto $L$ which composed with $\pi$ yield the given $\beta$, we get that $k \le \phi_L(m)/\card{\Gamma}\phi_{L/M}(m)$ and thus $\card{K} \le \phi_L(m)/\card{\Gamma}\phi_{L/M}(m)$.
\end{proof}