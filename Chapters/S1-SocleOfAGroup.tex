\section{Socle of a Group}

\begin{definition}
    The \textbf{socle} of a group $G$, henceforth denoted by $\mathbf{\soc G}$, is the subgroup generated by all its minimal subgroups.
\end{definition}

If $G$ has no minimal normal subgroups, then $\soc G = 1$. This is so because the group generated by the empty set is the trivial group.

The next Theorem is well-known and an alternative proof can be found in \cite[p.~87]{RobinsonCTG}. 
\begin{theorem}
    The socle of a finite group $H$ is a direct product of minimal normal subgroups.
\end{theorem}

\begin{proof}
    Let $M_1,...,M_k$ be the minimal normal subgroups of $H$. We know that $\soc{H}$ is the product of its minimal normal subgroups, that is $\soc{H} = M_1...M_k$.

    Now we will construct the following subsequence $$M_{i_1} = M_1, M_{i_2} = M_2, ..., M_{i_j}$$ where $M_{i_l} \cap (M_1...M_{i_{l-1}}) = 1$ for $1 \le l \le j$ and $M_i \le (M_1...M_{l})$ for all $1 \le i \le i_l$.

    Assuming we have $M_{i_l}$ we can choose $M_{i_{l+1}}$ in the following way: $i_{l+1}$ is the smallest number such that $i_{l+1} > i_l$ and $M_{i_{l+1}} \cap M_1...M_l = 1$; if no such number exists the subsequence is completed.

    We claim that a subsequence constructed in this way satisfies our conditions.
    
    Obviously if $i \le i_l$ we have by hypothesis 
    $$
    M_i \le M_{i_1}...M_{i_l} \le M_{i_1}...M_{i_l}M_{i_{l+1}}.
    $$ 
    If $i_l < i < i_{l+1}$ we have by the construction of $M_{i_{l+1}}$ that $M_i \cap M_{i_1}...M_{i_l} \ne 1$ and since $M_i \cap M_{i_1}...M_{i_l}$ is normal in $H$ it must be $M_i$. Hence $M_i \le M_{i_1}...M_{i_l} \le M_{i_1}...M_{i_{l+1}}$.

    It is thus clear that $\soc{H} = M_{i_1}...M_{i_j}$ and since $M_{i_l} \cap (M_1...M_{i_{l-1}}) = 1$ for $1 \le l \le j$ we have $\soc{H} = M_{i_1}\times ... \times M_{i_j}$.
\end{proof}

Our choice of the minimal normal subgroup $M_1$ in the last theorem was completely arbitrary, whence we can easily see that any minimal normal subgroup is complemented in $\soc{H}$.

 \begin{theorem}
    \label{th:SocC}
    Let $H$ be a finite group. Suppose that all its minimal normal subgroups are abelian and complemented. Then $\soc{H}$ is complemented.
\end{theorem}

\begin{proof}
Let $N_1,...,N_r$ be the minimal normal subgroups of $H$ and $C_1,...,C_r$ be its complements respectively. We are going to construct a complement for $\soc H$ starting from $C_1$.

Let $K_1$ be a subgroup of $H$ such that $(\soc H)K_1 = H$, say for example $K_1 = C_1$ since $H = N_1C_1 = (\soc H)C_1$.
We have that $\soc H$ is abelian as the minimal normal subgroups are abelian and $\soc H$ is the direct product of some of them.
Due to $K_1 \cap \soc H \nsub K_1$ and $K_1 \cap \soc H \nsub \soc H$, as $\soc H$ is abelian, $K_1 \cap \soc H \nsub H = (\soc H) K_1$.

If $K_1 \cap \soc H \ne 1$, since $K_1 \cap \soc H$ is normal it contains a minimal normal subgroup, $N_i$ say.
We assert that $K_1 = N_i(C_i \cap K_1)$. The first inclusion follows as for any $k_1 \in K_1 \subseteq N_iC_i$, $k_1 = n_ic_i$ for some $n_i \in N_i$, $c_i \in C_i$ and hence $c_i = n_i^{-1}k_1 \in K_1 \implies c_i \in K_1 \cap C_i$. The other inclusion is trivial. Hence we have
$$
H = (\soc H) K_1 = (\soc H) N_i(K_1 \cap C_i) = (\soc H) (K_1 \cap C_i)
$$
We thus proved that if there exists a group $K_1$ such that $(\soc H) K_1 = H$ and the intersection $\soc H \cap K_1$ is nontrivial then there exists another group $K_2 = K_1 \cap C_i$ such that $(\soc H) K_2 = H$ and $\soc H \cap K_2$ is strictly contained in $\soc H \cap K_1$. The inclusion of $\soc H \cap K_2$ in $\soc H \cap K_1$ is strict since the latter contains $N_i$ but by construction of $K_2$ the former does not. Proceeding in this manner we can construct a $K_j$ that complements $\soc H$.

\end{proof}