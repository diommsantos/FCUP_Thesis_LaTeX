\section{The M abelian case}

\newcommand{\dc}[0]{\dot{c}}

If $M$ is abelian then it is complemented by $C$ in $L$. Furthermore it was already proved that $\diag {C^k}$ complements $M^k$ in $L_k$. To simplify notation, given $l \in L$ we will denote by $\dl \in \diag{L^k}$ the element with all coordinates equal to $l$ and by $\dot{C} = \diag{C^k}$.

Now we can define the following group action:

\begin{align*}
    \cdot \colon L/M \times M^k &\rightarrow M^k \\
                (lM, m)  &\mapsto \dl^{-1}m\dl.
\end{align*}

We claim this action is well defined. Let $(l_1M, m_1) = (l_2M, m_2)$ where $l_1M, l_2M \in L/M$ and $m_1, m_2 \in M^k$.
Then $l_1^{-1}M = l_2^{-1}M$ and thus $\dot{l_1}^{-1} = \dot{l_2}^{-1}\dot{x}$ for some $x \in M$. We then obtain 
\begin{align*}
    &\dot{l_1}^{-1}m_1\dot{l_1} = \dot{l_2}^{-1}\dot{x}m_1\dot{x}^{-1}\dot{l_2} = \dot{l_2}^{-1}\dot{x}\dot{x}^{-1}m_1\dot{l_2} = \dot{l_2}^{-1}m_1\dot{l_2},
\end{align*}
where the third equality follows since $M$ is abelian.

Now let us consider
\begin{align*}
    \rho \colon &C \rightarrow L/M \\
                &c \mapsto cM.
\end{align*}
Obviously $\rho$ is well defined and is an homomorphism. It is also injective since for all $c_1, c_2 \in C$, such that $c_1M = c_2M$ then $c_2^{-1}c_1 \in M \cap C = 1$ and consequently $c_1 = c_2$. Furthermore by the First Isomorphism Theorem $\card{L/M} = \card{CM/M} = \card{C}\card{M}/\card{M} = \card{C} = \card{C/\ker rho} = \card{\rho(C)}$. Then since $\rho(C) \subseteq L/M$ we conclude that $\rho(C) = L/M$. In other words $\rho$ is a surjective function. We proved that $\rho$ is an isomorphism, and so we can define $\Psi = \rho^{-1} \colon L/M \rightarrow C$.  

Also, let $\beta \colon L_k \rightarrow L/M$ be a surjective homomorphism.

We have now all the necessary conditions to define the group action:

\begin{align*}
    \cdot \colon L/M \times M &\rightarrow M \\
                (lM, m)  &\mapsto \Psi(\beta(\dl))^{-1}m\Psi(\beta(\dl)).
\end{align*}

We claim that this function is well defined. 
Let $(l_1M, m_1) = (l_2M, m_2)$ where $l_1M, l_2M \in L/M$ and $m_1, m_2 \in M$. Then since $l_1M = l_2M \iff l_2^{-1}l_1 \in M$, $\dot{l_2}^{-1}\dot{l_1} \in M^k$ ($= \ker \beta$ by Theorem \ref{S4:kerbeta}), $\beta(\dot{l_2}^{-1}\dot{l_1}) = 1$. 
This in turn implies $ \beta(\dot{l_2})^{-1}\beta(\dot{l_1}) = 1 \implies \beta(\dot{l_1}) = \beta(\dot{l_2})$. 
Thus 
\begin{align*}
    l_1M \cdot m_1 = \Psi(\beta(\dot{l_1}))^{-1}m_1\Psi(\beta(\dot{l_1})) = \Psi(\beta(\dot{l_2}))^{-1}m_2\Psi(\beta(\dot{l_2})) = l_2M \cdot m_2.
\end{align*}

\begin{theorem}
    \label{S4:LMhom}
    Let us assume $M$ is abelian. Given a surjective homomorphism $\beta \colon L_k \rightarrow L/M$, the set $\mathscr{S}_\beta$ is identical to the set of kernels of surjective $L/M$-homomorphisms $\nu \colon M^k \rightarrow M$ with the above group actions.
\end{theorem}

\begin{proof}
    To prove the first inclusion, let $N \in \mathscr{S}_\beta$. Then there exists a surjective homomorphism $\varphi$ such that $\ker \varphi = N$ and $\pi_L \circ \varphi = \beta$. We will now prove that the restriction of $\varphi$ to $M^k$ is a $L/M$-homomorphism with kernel $N$.


    Since $\varphi$ is surjective we obtain that $\varphi|_{M^k}(\soc {L_k}) \subseteq \soc{L} = M$ by Theorem \ref{hommnsub}. 
    Furthermore since $\pi_L \circ \varphi = \beta$, $\ker \varphi$ is contained in $\ker \beta = M^k$. 
    We also claim that this inclusion is strict, since if it were otherwise then $\card{L} = \card{L_k}/\card{\ker \varphi} = \card{L_k}/\card{M^k} = \card{L}/\card{M}$ but this is a contradiction. 
    Thus $\varphi(M^k)$ is a non-trivial normal subgroup in $L$. Since $M$ is a minimal normal subgroup we conclude that $\varphi(M^k) = M$.

    That $\ker \varphi|_{M^k} = N$ is obvious, since $N = \ker \varphi \subseteq \ker \beta = M^k$.

    Now for any $lM \in L/M$ and any $m \in M^k$, we have that $\varphi(\dl)M = \pi_L \circ \varphi(\dl) = \beta(\dl) = \rho(\Psi(\beta(\dl))) = \Psi(\beta(\dl))M \implies \varphi(\dl)^{-1}M = \Psi(\beta(\dl))^{-1}M$. Thus there exists an $x \in M$ such that $\varphi(\dl)^{-1} = \Psi(\beta(\dl))^{-1}x^{-1} \implies \varphi(\dl) = x\Psi(\beta(\dl))$.
    It then follows that
    \begin{align*}
        \varphi|_{M^k}(lM \cdot m) &= \varphi(\dl^{-1} m \dl) \\
        &= \varphi(\dl)^{-1} \varphi(m) \varphi(\dl) \\
        &= \Psi(\beta(\dl))^{-1}x^{-1} \varphi(m) x\Psi(\beta(\dl)) \\
        &= \Psi(\beta(\dl))^{-1}x^{-1}x \varphi(m) \Psi(\beta(\dl)), \text{ since $M$ is abelian} \\
        &= lM \cdot \varphi|_{M^k}(m) 
    \end{align*}
    and the proof of this inclusion is complete.

    To prove the other inclusion, let $\nu$ be a $L/M$-homomorphism. 
    Let us first define
    \begin{align*}
        \psi \colon L_k &\rightarrow L \\
                \dc m &\mapsto \Psi(\beta(\dc))\nu(m),
    \end{align*}
    where $\dc \in \dot C$ and $m \in M^k$.
    This function is well-defined since $\dot{C}$ and $M^k$ are complements. It is a homomorphism since for any $\dc_1, \dc_2 \in \dot{C}$ and $m_1, m_2 \in M^k$,
    \begin{align*}
        \psi(\dc_1 m_1 \dc_2 m_2) &= \psi(\dc_1 \dc_2 m_1^{\dc_2^{-1}} m_2) \\
        &= \Psi(\beta(\dc_1 \dc_2)) \nu(m_1^{\dc_2^{-1}}) \nu(m_2) \\
        &= \Psi(\beta(\dc_1)) \Psi(\beta(\dc_2)) \nu(\dc_2M \cdot m_1) \nu(m_2) \\
        &= \Psi(\beta(\dc_1)) \Psi(\beta(\dc_2)) (\dc_2M \cdot \nu(m_1)) \nu(m_2) \\
        &= \Psi(\beta(\dc_1)) \Psi(\beta(\dc_2)) \nu(m_1)^{\Psi(\beta(\dc_2))^{-1}} \nu(m_2) \\
        &= \Psi(\beta(\dc_1)) \nu(m_1) \Psi(\beta(\dc_2)) \nu(m_2) \\
        &= \psi(\dc_1 m_1) \psi(\dc_2 m_2).
    \end{align*}
    We claim that $\ker \psi = \ker \nu$. Let $\dc \in \dot{C}$ and $m \in M^k$ be such that $\psi(\dc m) = 1$. Then
    $\Psi(\beta(\dc)) = \nu(m)^{-1} \in M \cap C = 1$. Thus $\dc \in \ker \beta \cap \dot{C} = M^{k} \cap \dot{C} = 1$ and $m \in \ker \nu$. What we conclude from this is that if $\psi(\dc m) = 1$ then $\dc = 1$ and $m \in \ker \psi$, that is $\ker \psi = \ker \nu$.

    Since $\psi(L_k) \subseteq L$, applying the First Isomorphism Theorem we obtain 
    $$
    \card{L} = \card{L_k}/\card{M}^{k-1} = \card{L_k}/\card{\ker \psi} = \card{\psi(L_k)}
    $$ 
    and thus surjectivity follows.
        
    We also easily verify that 
    \begin{align*}
        \pi_L \circ \psi(\dc m) &= \pi_L(\psi(\dc))\pi_L(\psi(m)) \\
        &= \pi_L(\Psi(\beta(\dc)))M \\
        &= \Psi(\beta(\dc))M \\
        &= \beta(\dc)M, \text{ since $\rho(\Psi(\beta(\dc))) = \beta(\dc) \implies \Psi(\beta(\dc))M = \beta(\dc)$} \\
        &= \beta(\dc)    
    \end{align*}
    
    \end{proof}

    \begin{theorem}
        \label{cardSA}
        Let us assume that $M$ is abelian. Given a surjective homomorphism $\beta \colon L_k \rightarrow L/M$, the cardinality of the set $\mathscr{S}_\beta$ is $k$ when $M$ is non-abelian; it is $(q^k-1)/(q-1)$ when $M$ is abelian and $q$ is the number of $(L/M)$-endomorphisms of $M$.
    \end{theorem}

    \begin{proof}
        If $M$ is abelian, by Theorem \ref{S4:LMhom} we have to count the kernels of surjective $(L/M)$-homomorphisms from $M^k$ to $M$. It is claimed on \cite[Lemma 2.5]{DallaVoltaFGNMGAPQ} that this number is $(q^k-1)/(q-1)$ where $q$ is the number of $(L/M)$-endomorphisms of $M$.
    \end{proof}




