\section{The M abelian case}

\newcommand{\dc}[0]{\dot{c}}

If $M$ is abelian then it is complemented by $C$ in $L$. Furthermore it was already proved that $\diag {C^k}$ complements $M^k$ in $L_k$. To simplify notation, given $l \in L$ we will denote by $\dl \in \diag{L^k}$ the element with all coordinates equal to $l$ and by $\dot{C} = \diag{C^k}$.

Now we can define the following group action:

\begin{align*}
    \cdot \colon L/M \times M^k &\rightarrow M^k \\
                (lM, m)  &\mapsto \dl^{-1}m\dl.
\end{align*}

We claim this action is well defined. Let $(l_1M, m_1) = (l_2M, m_2)$ where $l_1M, l_2M \in L/M$ and $m_1, m_2 \in M^k$.
Then $l_1^{-1} = l_2^{-1}x$ for some $x \in M$ and thus $\dot{l_1}^{-1} = \dot{l_2}^{-1}\dot{x}$. We then obtain 
\begin{align*}
    &\dot{l_1}^{-1}m_1\dot{l_1} = \dot{l_2}^{-1}\dot{x}m_1\dot{x}^{-1}\dot{l_2} = \dot{l_2}^{-1}\dot{x}\dot{x}^{-1}m_1\dot{l_2} = \dot{l_2}^{-1}m_1\dot{l_2},
\end{align*}
where the third equality follows since $M$ is abelian.

Now let us set $x_M = 1 \in L$ and choose for any other $lM \in L/M$ one $x_{lM} \in L$ such that $x_{lM}M = lM$. We can thus define the function
\begin{align*}
    \Psi \colon &L/M \rightarrow L \\
                &lM \mapsto x_{lM}.
\end{align*}
Let us note that obviously for any $lM \in L/M$, $lM = \pi_L(\Psi(lM))$ and $\Psi(l_1)\Psi(l_2)M = \Psi(l_1)M\Psi(l_2)M = l_1Ml_2M = l_1l_2M = \Psi(l_1l_2)M$. Also, let $\beta \colon L_k \rightarrow L/M$ be a surjective homomorphism.

We have now all the necessary conditions to define the group action:

\begin{align*}
    \cdot \colon L/M \times M &\rightarrow M \\
                (lM, m)  &\mapsto \Psi(\beta(\dl))^{-1}m\Psi(\beta(\dl)).
\end{align*}

We claim that this function is well defined. 
Let $(l_1M, m_1) = (l_2M, m_2)$ where $l_1M, l_2M \in L/M$ and $m_1, m_2 \in M$. Then since $l_1M = l_2M \iff l_2^{-1}l_1 \in M$, $\dot{l_2}^{-1}\dot{l_1} \in M^k$ and thus $\beta(\dot{l_2}^{-1}\dot{l_1}) = 1$. 
This in turn implies $ \beta(\dot{l_2})^{-1}\beta(\dot{l_1}) = 1 \implies \beta(\dot{l_1}) = \beta(\dot{l_2})$. 
Thus: 

\begin{align*}
    l_1M \cdot m_1 = \Psi(\beta(\dot{l_1}))^{-1}m_1\Psi(\beta(\dot{l_1})) = \Psi(\beta(\dot{l_2}))^{-1}m_2\Psi(\beta(\dot{l_2})) = l_2M \cdot m_2.
\end{align*}

Furthermore we claim this function is a group action. We have that 
$$
1M \cdot m_1 = \Psi(\beta(1))^{-1}m_1\Psi(\beta(1)) = \Psi(M)^{-1}m_1\Psi(M) = m_1 
$$
where the third equality follows since $\Psi(M) = 1$.

Also since $(\Psi(\beta(\dot{l_1}))\Psi(\beta(\dot{l_2}))M^{-1}) = (\Psi(\beta(\dot{l_1})\beta(\dot{l_2}))M)^{-1} = (\Psi(\beta(\dot{l_1}\dot{l_2}))M)^{-1}$ there exists an $x \in M$ such that 
$$
(\Psi(\beta(\dot{l_1}))\Psi(\beta(\dot{l_2})))^{-1} = \Psi(\beta(\dot{l_1}\dot{l_2}))^{-1}x^{-1} \iff \Psi(\beta(\dot{l_1}))\Psi(\beta(\dot{l_2})) = x\Psi(\beta(\dot{l_1}\dot{l_2})).
$$
It thus follows that
\begin{align*}
    l_2M \cdot (l_1M \cdot m_1) &= \Psi(\beta(\dot{l_2}))^{-1}\Psi(\beta(\dot{l_1}))^{-1}m_1\Psi(\beta(\dot{l_1}))\Psi(\beta(\dot{l_2})) \\
    &= (\Psi(\beta(\dot{l_1}))\Psi(\beta(\dot{l_2})))^{-1}m_1\Psi(\beta(\dot{l_1}))\Psi(\beta(\dot{l_2})) \\
    &= \Psi(\beta(\dot{l_1}\dot{l_2}))^{-1}x^{-1}m_1x\Psi(\beta(\dot{l_1}\dot{l_2})) \\
    &= \Psi(\beta(\dot{l_1}\dot{l_2}))^{-1}m_1\Psi(\beta(\dot{l_1}\dot{l_2})), \text{ since $M^k$ is abelian} \\
    &= l_1l_2M \cdot m_1.
\end{align*}
The proof that this function is indeed an action is complete.


\begin{theorem}
    \label{S4:LMhom}
    Let us assume $M$ is abelian. Given a surjective homomorphism $\beta \colon L_k \rightarrow L/M$, the set $\mathscr{S}_\beta$ is identical to the set of kernels of surjective $L/M$-homomorphisms $\nu \colon M^k \rightarrow M$ with the above group actions.
\end{theorem}

\begin{proof}
    To prove the first inclusion, let $N \in \mathscr{S}_\beta$. Then there exists a surjective homomorphism $\varphi$ such that $\ker \varphi = N$ and $\pi_L \circ \varphi = \beta$. We will now prove that the restriction of $\varphi$ to $M^k$ is a $L/M$-homomorphism with kernel $N$.


    Since $\varphi$ is surjective we obtain that $\varphi|_{M^k}(\soc {L_k}) \subseteq \soc{L} = M$ by Theorem \ref{hommnsub}. 
    Furthermore since $\pi_L \circ \varphi = \beta$, $\ker \varphi$ is contained in $\ker \beta = M^k$. 
    We also claim that this inclusion is strict, since if it were otherwise then $\card{L} = \card{L_k}/\card{\ker \varphi} = \card{L_k}/\card{M^k} = \card{L}/\card{M}$ but this is a contradiction. 
    Thus $\varphi(M^k)$ is a non-trivial normal subgroup in $L$. Since $M$ is a minimal normal subgroup we conclude that $\varphi(M^k) = M$.

    That $\ker \varphi|_{M^k} = N$ is obvious, since $N = \ker \varphi \subseteq \ker \beta = M^k$.

    Now for any $lM \in L/M$ and any $m \in M^k$, we have that $\varphi(\dl)M = \pi_L \circ \varphi(\dl) = \beta(\dl) = \Psi(\beta(\dl))M$ and thus there exist an $x \in M$ such that $\varphi(\dl) = x\Psi(\beta(\dl))$.
    Then follows that
    \begin{align*}
        \varphi|_{M^k}(lM \cdot m) &= \varphi(\dl^{-1} m \dl) \\
                                    &= \varphi(\dl)^{-1} \varphi(m) \varphi(\dl) \\
                                    &= \Psi(\beta(\dl))^{-1}x^{-1} \varphi(m) x\Psi(\beta(\dl)) \\
                                    &= \Psi(\beta(\dl))^{-1} \varphi(m) \Psi(\beta(\dl)), \text{ since $M$ is abelian} \\
                                    &= lM \cdot \varphi|_{M^k}(m) 
    \end{align*}
    and the proof of this inclusion is complete.

    To prove the other inclusion, let $\nu$ be a $L/M$-homomorphism. 
    Let us first define
    \begin{align*}
        \psi \colon L_k &\rightarrow L \\
                \dc m &\mapsto \Psi(\beta(\dc))\nu(m),
    \end{align*}
    where $\dc \in \dot C$ and $m \in M^k$.
    This function is well-defined since $\dot{C}$ and $M^k$ are complements. It is a homomorphism since for any $\dc_1, \dc_2 \in \dot{C}$ and $m_1, m_2 \in M^k$,
    \begin{align*}
        \psi(\dc_1 m_1 \dc_2 m_2) &= \psi(\dc_1 \dc_2 m_1^{\dc_2^{-1}} m_2) \\
                                   &= \Psi(\beta(\dc_1 \dc_2)) \nu(m_1^{\dc_2^{-1}}) \nu(m_2) \\
                                   &= \Psi(\beta(\dc_1)) \Psi(\beta(\dc_2)) \nu(\dc_2 \cdot m_1) \nu(m_2) \\
                                   &= \Psi(\beta(\dc_1)) \Psi(\beta(\dc_2)) (\dc_2 \cdot \nu(m_1)) \nu(m_2) \\
                                   &= \Psi(\beta(\dc_1)) \Psi(\beta(\dc_2)) \nu(m_1)^{\Psi(\beta(\dc_2))^{-1}} \nu(m_2) \\
                                   &= \Psi(\beta(\dc_1)) \nu(m_1) \Psi(\beta(\dc_2)) \nu(m_2) \\
                                   &= \psi(\dc_1 m_1) \psi(\dc_2 m_2).
    \end{align*}
    We claim $\ker \psi = \ker \nu$. Let $\dc \in \dot{C}$ and $m \in M^k$ be such that $\psi(\dc m) = 1$. Then
    $\Psi(\beta(\dc)) = \nu(m)^{-1} \in M \iff \beta(\dc) = \Psi(\beta(\dc))M = M$. Thus $\dc \in \ker \beta \cap \dot{C} = M^{k} \cap \dot{C} = 1$. We thus obtain that
    $\Psi(\beta(\dc)) = \Psi(M) = 1$ and thus $\nu(m) = \Psi(\beta(\dc))\nu(m) = 1$. What we conclude from this is that if $\psi(\dc m) = 1$ then $\dc = 1$ and $m \in \ker \psi$, that is $\ker \psi = \ker \nu$.
    Surjectivity easily follows once we note that $\card{L_k}/\card{N} = \card{L_k}/\card{\ker \nu} = L$. Then since $\psi(L_k) \subseteq L$, applying the First Isomorphism Theorem we obtain $\card{L} = \card{L_k}/\card{N} = \card{L_k}/\card{\ker \psi} = \card{\psi(L_k)}$ and thus surjectivity.
        
    We also easily verify that $\pi_L \circ \psi(\dc m) = \pi_L(\psi(\dc))\pi_L(\psi(m))=\pi_L(\Psi(\beta(\dc))) = \beta(\dc)M = \beta(\dc)$.
    \end{proof}

    \begin{theorem}
        \label{cardSA}
        Let us assume that $M$ is abelian. Given a surjective homomorphism $\beta \colon L_k \rightarrow L/M$, the cardinality of the set $\mathscr{S}_\beta$ is $k$ when $M$ is non-abelian; it is $(q^k-1)/(q-1)$ when $M$ is abelian and $q$ is the number of $(L/M)$-endomorphisms of $M$.
    \end{theorem}

    \begin{proof}
        If $M$ is abelian, by Theorem \ref{S4:LMhom} we have to count the kernels of surjective $(L/M)$-homomorphisms from $M^k$ to $M$. It is claimed on \cite[Lemma 2.5]{DallaVoltaFGNMGAPQ} that this number is $(q^k-1)/(q-1)$ where $q$ is the number of $(L/M)$-endomorphisms of $M$.
    \end{proof}




