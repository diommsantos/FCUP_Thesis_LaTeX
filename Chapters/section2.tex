\chapter{The groups \texorpdfstring{$L_k$}{Lk}} \label{TheGroupsLk}

In this section $L$ will always denote a finite group with a unique minimal normal subgroup $M$. Furthermore if $M$ is abelian, we can also assume that $M$ is complemented in $L$.

Given a positive integer $k$ we will denote by $L^{k}$ the $k$-fold direct power of $L$ and by $\diag{L^{k}}$ the subgroup $\{(l_1,...,l_k) \in L^k | l_1 = l_2 = ... = l_k\}$. If not explicitly said otherwise, we will assume throughout this section that $k$ is a positive integer.

Also $\pi_i$ will denote the projection of the $i$-th coordinate from $L^k$ onto $L$ and $\Mi = 1\times ... \times M \times ... \times 1$ the subgroup of $L^k$ whose elements have some $m \in M$ for the $i$-th coordinate and $1$ for the rest.

\begin{definition}
    Given a positive integer $k$, the group $L_k$ is a subgroup of $L^{k}$ defined by:
    $$
    L_k = \{ (l_1,...,l_k) \in L^k | l_1M = ... = l_kM \}.
    $$
\end{definition}

Let $k$ be a positive integer.
We will often simply write $\pi_i$ to denote $\pi_i|_{L_k}$, the restriction of $\pi_i$ to $L_k$. One property of the functions $\pi_i|_{L_k}$ is that given $S \subseteq L$,
$$
\pi_i|_{L_k}^{-1}(S) = \pi_i^{-1}(S) \cap L_k.
$$
Thus it follows that $\pi_i|_{L_k}^{-1}(M) = \pi_i^{-1}(M) \cap L_k = (L \times ... L \times M \times L .... \times L) \cap L_k = M^k$.
\section{Some basic properties}
The properties of the groups \(L_k\) will be referenced implicitly throughout.


\begin{theorem}
    \label{altdscLk}
     The group $L_k$ can be described as $\diag{L^{k}}M^{k}$.
\end{theorem}

\begin{proof}
    For any $(l_1,...,l_k) \in L_k$, we have that for any $1 \le i \le k$, $l_1M = l_iM$. Hence it follows $l_i = l_1m_i$ for some $m_i \in M$. We thus have $(l_1,...,l_k) = (l_1,l_1,...,l_1)(1,m_2,...,m_k)$ as pretended. 
    
    For the other inclusion it suffices to notice that $M^k, \diag{L^k} \subseteq L_k$, hence $\diag{L^k}M^k \subseteq L_k$.
\end{proof}

\begin{theorem}
    The socle of $L_k$ is $M^k$.
\end{theorem}

\begin{proof}
    We have by Theorem \ref{hommnsub} that: if $N$ is a minimal normal subgroup of $L_k$ then for all $1 \le i \le k$, $\pi_i(N)$ is either equal to $1$ or a minimal normal subgroup of $L$. Since $L$ has a unique minimal normal subgroup $\pi_i(N) = 1$ or $\pi_i(N) = M$.
    
    We thus have $N \subseteq \bigcap_{i=1}^{k} \pi_i|_{L_k}^{-1}(M) = M^{k}$. Since the last inclusion holds for any minimal normal subgroup $N$, it easily follows that $\soc{L_k} \subseteq M^k$.

    Since the groups $\Mi$ are minimal normal subgroups of $L_k$, we obviously have $M^k=\Mi[1]...\Mi[k] \subseteq \soc{L_k}$.
\end{proof}

\begin{theorem}
    \label{qtLksoc}
    The group $L_k/M^k$ is isomorphic to $L/M$.
\end{theorem}

\begin{proof}
    We have that:
    $$
        \frac{L_k}{M^k}=\frac{\diag{L^k}M^k}{M^k} \cong \frac{\diag{L^k}}{\diag{L^k} \cap M^k}
    $$
    by the Second Isomorphism Theorem and by Theorem \ref{altdscLk}. 
    
    Evidently, $\diag{L_k} \cap M^k = \diag{M^k}$ and thus is easy to verify that $\diag{L^k}/\diag{M^k} \cong L/M$.
\end{proof}

\begin{theorem}
    If $M$ is abelian and complemented by $C$ in $L$, then $M^k$ is complemented by $\diag{C^k}$ in $L_k$.
\end{theorem}

\begin{proof}
    We have that $\diag{L^k} = \diag{(CM)^k} = \diag{C^k}\diag{M^k}$. 
    Then by Theorem \ref{altdscLk}, $L_k = \diag{L^k}M^k = \diag{C^k}\diag{M^k}M^k = \diag{C^k}M^k$. 
    Furthermore for all $x \in \diag{C^k} \cap M^k$ and all $1 \le i \le k$, $\pi_i(x) \in \pi_i(\diag{C^k}) \cap \pi_i(M^k) = C \cap M = 1$. 
    This means that all the coordinates of $x$ are $1$ and consequently that $x = 1$. The proof is thus complete.
\end{proof}


\section{The minimal normal subgroups of \texorpdfstring{$L_k$}{Lk}}

\begin{theorem}
    \label{nabmnsub}
    If $M$ is not abelian, any minimal normal subgroup of $L_k$ is of the form $\Mi$ for some $i$.
\end{theorem}

\begin{proof}
    Let $N$ be a minimal normal subgroup of $L_k$.
    Once again by Theorem \ref{hommnsub}, for each $1 \le i \le k$, $\pi_i(N)$ is either $1$ or $M$. We also have that $\pi_j(N) = M$ for some $j$ otherwise $N$ would be the trivial subgroup which is a contradiction.

    Let $j$ be such that $\pi_j(N) = M$. Now we will prove that for any $1 \le i \le k$ different of $j$, $\pi_i(N) = 1$ which gives us the result.
    
    Suppose on the contrary that exists some $1 \le i \le k$ different of $j$ such that $\pi_i(N) = M$.
    By Theorem \ref{mnsubsc} we obtain $[N, \Mi] = 1$. Since $M$ is not abelian we can choose elements $m_1, m_2 \in M$ such that $m_1m_2 \ne m_2m_1$. Besides we have by hypothesis that $m_1 = \pi_i(n_1)$ and $m_2 = \pi_i(n_2)$ for some $n_1 \in N$ and some  $n_2 \in M_i$. It thus follows that:
    \begin{align*}
       m_1m_2 &= \pi_i(n_1)\pi_i(n_2) \\
              &= \pi_i(n_1n_2) \\
              &= \pi_i(n_2n_1) \text{, by } [N, \Mi] = 1 \\
              &= \pi_i(n_2)\pi_i(n_1) = m_2m_1 \text{, a contradiction.}
    \end{align*}

\end{proof}

With the previous theorem, the minimal normal subgroups in the case where $M$ is non abelian are fully characterized.
What can we say about the minimal normal subgroups of $L_k$ in the abelian case? The next theorems provide us some nice properties.

\begin{theorem}
    \label{abmnsub}
    Let $N$ be a minimal normal subgroup of $L_k$. Then $N$ has order $\card{M}$. 
    
    %Furthermore if for some $1 \le i \le k$, $\pi_i(N) = M$ and an element $n$ of $N$ has $i$-th coordinate equal to $1$ then all its coordinates are $1$, that is if $\pi_i(n) = 1$ for some $1 \le n \le k$ then $n = 1$.
\end{theorem}

\begin{proof}
    We have once more that $\pi_i(N) = M$ for some $1 \le i \le k$. Considering the appropriate restriction of $\pi_i$, by the First Isomorphism Theorem we get that $\card{N} = \card{M} \cdot \card{\ker{\pi_i|_{N}}}$ which implies $\card{N} \geq \card{M}$.

    Let's assume that $\card{N} > \card{M}$. Then there is some $n \in \ker{\pi_i|_{N}}$ with not all coordinates $1$ (obviously the $i$-th coordinate must be $1$).

    Consider $\subgen{n}_{L_k}$ contained in $\ker{\pi_i|_{N}}$. Since $n \in N$, and $\subgen{n}_{L_k}$ is the smallest normal subgroup generated by $n$ we must have $\subgen{n}_{L_k} \subseteq N$. Since $\pi_i(\subgen{n}_{L_k}) = \subgen{\pi_i(n)}_{L_k} = 1$ and not all elements of $N$ are in $\ker{\pi_i|_{N}}$ we obtain that $\subgen{n}_{L_k}$ is a non-trivial group strictly contained in $N$. This is a contradiction since $N$ is a minimal normal subgroup.
\end{proof}

\begin{definition}
    Given $l \in L$ and a positive integer $q$, let $\dl$ denote the element of $\diag{L^q}$ with all coordinates equal to $l$.
\end{definition}

The previous definition depends on which power of the group $L$ we are considering, and often the only way to decide the ambient group is through context, but what we lose in formal rigor we gain in readability. Such an example of improved readability is the statement of the next theorem.

\begin{theorem}
    \label{cplNsoc}
    Let $N$ be a minimal normal subgroup of $L_k$. Then there exists a complement $C \nsub L_k$ of $N$ in $\soc{L_k}$ and an isomorphism $\phi \colon C \rightarrow M^{k-1}$ that satisfies $\phi(c^{\dl}) = \phi(c)^{\dl}$ for any $c \in C$ and $l \in L$.
\end{theorem}

\begin{proof}
    We have once more that $\pi_i(N) = M$ for some $1 \le i \le k$. 
    
    Let $C = \Mi[1]...\Mi[i-1]\Mi[i+1]...\Mi[k]$. We claim that $N \cap C = 1$. Let's note first that any element of $N \cap C \subseteq C$ has $i$-th coordinate $1$. Thus $N \cap C \nsub L_k$ is strictly contained in the minimal normal subgroup $N$. So it must be $1$.

    From the Second Isomorphism Theorem we get that $\card{NC}/\card{N} = \card{C} / \card{N \cap C}$.
    We obtain $\card{NC} = \card{N}\card{C}\cdot 1$ and since $\card{N} = \card{M}$ by Theorem \ref{abmnsub}, $\card{NC} = \card{M}\card{C} = \card{M}^k = \card{\soc{L_k}}$.
    Obviously $NC \subseteq \soc{L_k}$ as $N,C \subseteq \soc{L_k}$. We thus conclude that $NC = \soc{L_k}$.
    

    The expected isomorphism
    \begin{align*}
        \phi \colon C = M \times ... \times M \times &1 \times M \times ... \times M \longrightarrow M^{k-1} \\
        (m_1,...,m_i,&1,m_{i+1},...,m_{k}) \mapsto (m_1,...,m_i,m_{i+1},...,m_{k})
    \end{align*}
    works. 
    
    The property is easily verified since for any $l \in L$, $(m_1,...,m_i,1,m_{i+1},...,m_{k}) \in C$:
    \begin{align*}
        &\phi(lm_1l^{-1},...,lm_il^{-1},l1l^{-1},lm_{i+1}l^{-1},...,lm_{k}l^{-1}) &= \\
        &(lm_1l^{-1},...,lm_il^{-1},lm_{i+1}l^{-1},...,lm_{k}l^{-1}) &= \\
        &(l,...,l)(m_1,...,m_i,m_{i+1},...,m_{k})(l^{-1},...,l^{-1}) &= \\ 
        &(l,...,l)\phi(m_1,...,m_i,1,m_{i+1},...,m_{k})(l,...,l)^{-1}
    \end{align*}
\end{proof}


\section{Normal subgroups and quotients of \texorpdfstring{$L_k$}{Lk}}

\begin{theorem}
    \label{S2:QLkmnsub}
    The quotient of $L_{k+1}$ by any of its minimal normal subgroups is isomorphic to $L_k$.
\end{theorem}

\begin{proof}
    Let $N$ be a minimal normal subgroup of $L_{k+1}$.

    If $M$ is not abelian, by Theorem \ref{nabmnsub}, $N = \Mi$ for some $1 \le i \le k+1$. It is trivial to verify that
    \begin{align*}
        \psi \colon &L_{k+1} \longrightarrow L_k \\
                (l_1,.&..,l_{k+1}) \mapsto (l_1,...,l_{i-1},l_{i+1},...,l_{k+1})
    \end{align*}
    is a surjective homomorphism. We will now verify that $\ker \psi = \Mi = N$. Obviously $\Mi \subseteq \ker \psi$ as the $i$-th coordinate is "forgotten" and that is the only non $1$ coordinate in $\Mi$. For the other inclusion, we first have by Theorem \ref{qtLksoc} that $\card{L_k} = \card{L/M} \card{M}^k$. By the First Isomorphism Theorem, $\card{L_{k+1}}/\card{\ker{\psi}} = \card{L_k}$. From this two equalities follows that $\card{\ker \psi} = \card{\Mi} = \card \Mi$. Therefore $\ker \psi = \Mi = N$ and consequently
    $L_{k+1}/N = L_{k+1}/\ker \psi \cong L_k$ by the First Isomorphism Theorem.

    If $M$ is abelian, by hypothesis we have that $M$ is complemented in $L$ say by $C_L$. By Theorem \ref{S2:complMkab}, $\diag{C_L^q}$ is a complement of $M^q$ in $L_q$.

    By Theorem \ref{cplNsoc}, we have that $N$ is complemented in $M^{k+1}$ and that its complement, say $C_{\soc{L}}$, is normal in $L_{k+1}$ and isomorphic to $M^{k}$. 
    As $\diag{C_L^{k+1}}$ complements $M^{k+1}$, by Theorem \ref{smdptrans} we have that $N$ is complemented in $L_{k+1}$ by $C=\diag{C_L^{k+1}}C_{\soc L_{k+1}}$.

    Then we obtain:
    \begin{align*}
        L_{k+1}/N = CN/N \cong C/(C \cap N) = C/1 \cong C,
    \end{align*}
    using the Second Isomorphism Theorem.

    Now we need to show that $C = \diag{C_L^{k+1}}C_{\soc L_{k+1}} \cong L_k$. Let $\psi$ be the function defined by
    \begin{align*}
        \psi \colon C & \rightarrow L_k=\diag{C_L^k}M^k \\
                    \dl s &\mapsto \dl \phi(s)
    \end{align*}
    where $l \in C_L^{k+1}$ and $s \in C_{\soc{L_{k+1}}}$ and $\phi$ is the isomorphism from Theorem \ref{cplNsoc}.

    Let $l_1, l_2 \in C_L$ and $s_1, s_2 \in C_{\soc{L_{k+1}}}$.
    To see that $\psi$ is well defined it suffices to notice that if $\dl_1s_1 = \dl_2s_2$ then:
    \begin{align*}
        &\dl_2^{-1}\dl_1 = s_2s_1^{-1} \in C_L^{k+1} \cap C_{\soc{L_{k+1}}} = 1 \\
        \implies & \dl_2 = \dl_1, s_1 = s_2 \\
        \implies & \dl_1\phi(s_1) = \dl_2\phi(s_2)
    \end{align*}

    Knowing that $\phi$ is an isomorphism, is easy to see that $\psi$ is surjective since every element $\dl m \in L_k$ (where $\dl \in \diag{C_L^k}$ and $m \in M^k$) is the image by $\psi$ of $\dl\phi^{-1}(m)$.

    Injectivity follows comparing the orders of $C$ and $L_k$. We have
    $$
    \card{C} = \card{L_{k+1}} / \card{N} = \card{L_k}
    $$
    remembering that $C$ complements $N$ in $L_{k+1}$ and that $\card{N} = \card{M}$.
    We thus conclude that $\psi$ is a bijection, since it is a surjection between finite groups of equal orders.

    Now we need only to check that $\psi$ is a homomorphism, which follows from:
    \begin{align*}
        \psi(\dl_1s_1\dl_2s_2) &= 
    \psi(\dl_1\dl_2s_1^{\dl_2^{-1}}s_2)\\
    &= \dl_1\dl_2\phi(s_1^{\dl_2^{-1}}s_2)\\
    &=\dl_1\dl_2\phi(s_1^{\dl_2^{-1}})\phi(s_2)\\
    &=\dl_1\dl_2\phi(s_1)^{\dl_2^{-1}}\phi(s_2)\\
    &=\dl_1\phi(s_1)\dl_2\phi(s_2)\\
    &= \psi(\dl_1s_1)\psi(\dl_2s_2).
    \end{align*}
\end{proof}

\begin{theorem}
    \label{th:nsubsoc}
    Let $N$ be a normal group of $L_k$. Then either $\soc {L_k} \le N$ or $N \le \soc{L_k}$.
\end{theorem}

\begin{proof}
    The proof will be done by induction on $k$.

    Since $L$ has a unique minimal normal subgroup the proposition is easily seen to be true for $k=1$.
    
    Suppose now that the result holds for $k$ and that $N$ is not a subgroup of $\soc{L_{k+1}}$. Then there exists a minimal normal subgroup $U$ of $L_{k+1}$ such that $U \subseteq N$.
    We have that $L_{k+1}/U \cong L_{k}$ by some isomorphism $f$.
    By induction either $\soc{L_{k}} \le f(N/U)$ or $f(N/U) \le \soc{L_k}$. Since $N$ is not a subgroup of $\soc{L_{k+1}}$ we have that $N/U \nleqslant \soc{L_{k+1}}/U$ which implies that $\soc{L_k} = f(\soc{L_{k+1}/U}) \le f(N/U)$ and thus the first case holds.
    
    Applying $f^{-1}$ to $\soc {L_k} \le f(N/U)$ we get $\soc{L_{k+1}}/U \subseteq \soc{L_{k+1}/U} \le N/U$. Hence as $U \subseteq \soc{L_{k+1}}, N$ we obtain $\soc{L_{k+1}} = \pi^{-1}(\soc{L_{k+1}/U}) \le \pi^{-1}(N/U) = N$, where $\pi: L_k \rightarrow L_k/U$ is the usual projection.
\end{proof}

\section{The Sequence \texorpdfstring{$d(L_k)_{k \in \mathbb{N}}$}{dLkN}}

Throughout we asume that $\mathbb{N}$ is the set of positive integers, i.e does not contain $0$.
The sequence $d(L^k)_{k \in \mathbb{N}}$ is called the \textit{growth sequence} and has been studied in \cite{WiegoldGSFG, WiegoldGSFGII, WiegoldGSFGIII, WiegoldGSFGIV}. Here we study the sequence $d(L_k)_{k \in \mathbb{N}}$ given the importance that the groups $L_k$ assume in the study of the minimal number of generators of finite groups. 

\begin{theorem}
    \label{S2:nondecLk}
    The sequence $d(L_k)_{k \in \mathbb{N}}$ is non-decreasing.
\end{theorem}

\begin{proof}
    Let us assume to obtain a contradiction that there is some positive integer $k$ such that $d(L_{k+1}) < d(L_k)$. 
    
    Let us consider the function $\rho: L_{k+1} \rightarrow L_k$ that drops the last coordinate. That is given $(x_1,...,x_{k+1}) \in L_{k+1}$, $\rho((x_1,...,x_{k+1})) = (x_1,...,x_k)$.

    Now let $l_1,...,l_m$ be a minimal generating set of $L_{k+1}$, that is $
    \subgen{l_1,...,l_m} = L_{k+1}$ and $d(L_{k+1}) = m$.
    Obviously $\rho$ is surjective and thus we obtain that 
    $$
    L_k = \rho(L_{k+1}) = \rho(\subgen{l_1,...,l_m}) = \subgen{\rho(l_1),...,\rho(l_m)}.
    $$
    This is of course a contradiction since we assumed that $m = d(L_{k+1}) < d(L_k)$.
\end{proof}

\begin{theorem}
    \label{S2:bounddLk}
    For all positive integers $k$, $d(L_{k+1}) \le d(L_k) + 1$.    
\end{theorem}
\begin{proof}
    Let $d(L_k) = d$, then there are $l_1, l_2, ... l_d \in L_k$ such that $\subgen{l_1, l_2, ... l_d} = L_k$. By abuse of notation, given $l \in L_k$ let $\ell \in L_{k+1}$ be the $k+1$-tuples whose first $k$ coordinates are the coordinates of $l$ and whose last two coordinates are equal, that is $\pi_k(\ell) = \pi_{k+1}(\ell)$. 
        
    Let $m \in 1 \times ... \times 1 \times M \le L_{k+1}$ and consider
    $$H = \subgen{m^\ell | l \in L_k} \le L_{k+1}.$$
    For all $i \in \left\{1, ..., k \right\}$, $\pi_i(H) = 1$, because
    $$\pi_i(\subgen{m^\ell | l \in L_k}) = \subgen{\pi_i(m)^{\pi_i(\ell)} | l \in L_k} = \subgen{1^{\pi_i(\ell)} | l \in L_k} = 1.$$ 
    And $\pi_{k+1}(H) = M$ since
    $$\pi_{k+1}(\subgen{m^\ell | l \in L_k}) = \subgen{\pi_{k+1}(m)^{\pi_{k+1}(\ell)} | l \in L_k} = \subgen{\pi_{k+1}(m)}_L,$$
    where the last inequality follows since $\pi_{k+1}(l_1) = l_1, \pi_{k+1}(l_2) = l_2, ..., \pi_{k+1}(l_d) = l_d$ and $\subgen{l_1, l_2, ... l_d} = L_k$. We thus have that $\pi_{k+1}(H)$ is a nontrivial normal subgroup of $L$ that is contained in $M$, thus $\pi_{k+1}(H) = M$ due to the minimality of $M$. We thus have that $H = 1 \times ... \times 1 \times M$.
    Now $L_{k+1} = \left\{ \ell | l \in L_{k+1} \right\}(1 \times ... \times 1 \times M) = \left\{ \ell | l \in L_{k+1} \right\} H = \subgen{\ell_1, \ell_2, ... , \ell_d, m}$
    and the result is proved.
    
\end{proof}

\begin{theorem}
    \label{S2:undLk}
    The sequence $d(L_1),...,d(L_k), ....$ is unlimited.
\end{theorem}

\begin{proof}
    Suppose for the sake of obtaining a contradiction that there exists a natural number $m$ such that $d(L_k)<m$ for all $k \in \mathbb{N}$.
    Let $F$ be a free group of rank $m$ and $K$ be the set of positive integers $k$ such that there exists a surjective homomorphism from $F$ to $L_k$. 
    It will be proved that $K$ is a finite set, a contradiction from which the result will follow.

    Let $k > 1 \in K$, then there exists a surjective homomorphism $\Psi : F \rightarrow L_k$. For $1 \le i \le k$, let $\gamma_i = \pi_i \circ \Psi : F \rightarrow L_k \rightarrow L$ and let $\pi : L \rightarrow L/M$ be the natural projection. For all $x \in F$, $\Psi(x) = (l_1,...,l_k)$ for some $(l_1,...,l_k) \in L_k$ thus:
    \begin{align*}
        &Ml_1 = Ml_2 = ... = Ml_k \text{ and } \\
        &\pi\gamma_1(x) = Ml_1, .... , \pi\gamma_k(x) = Ml_k \\
        \implies &\pi\gamma_1 = \pi\gamma_2 = ... = \pi\gamma_k.
    \end{align*}
    Let $i_1, i_2 \in \left\{ 1, ..., k \right\}$ and let $(m_1,..., m_k) \in M_k \le L_k$ with $m_{i_1} = 1, m_{i_2} \ne 1.$ Since $\Psi : F \rightarrow L_k$ is surjective there exists an $x \in F$ such that $\Psi(x) = (m_1,..., m_k)$. 
    We then have that $\gamma_{i_1}(x) = m_{i_1} = 1$ and $\gamma_{i_2}(x) = m_{i2} \ne 1$, and thus $x \in \ker \gamma_{i_1}$ and $x \notin \ker \gamma_{i_2}$ and hence $\ker \gamma_{i_1} \ne \ker \gamma_{i_2}$. We thus have that $\card{\left\{ \ker \gamma_{1}, ..., \ker \gamma_{k} \right\}} = k$. 
    
    By Theorem \ref{PHallT}, taking $\beta$ as $\pi\gamma_i$ for any $i$ (it was proved that this functions were all equal) and seeing $\left\{ \ker \gamma_{1}, ..., \ker \gamma_{k} \right\}$ as a subset of the appropriate $R$, the set of normal subgroups $N$ of $F$ arising as kernels of those homomorphisms of $F$ onto $L$ which composed with $\pi$ yield the given $\beta$, we get that $k \le \phi_L(m)/\card{\Gamma}\phi_{L/M}(m)$ and thus $\card{K} \le \phi_L(m)/\card{\Gamma}\phi_{L/M}(m)$.
\end{proof}